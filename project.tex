% Options for packages loaded elsewhere
\PassOptionsToPackage{unicode}{hyperref}
\PassOptionsToPackage{hyphens}{url}
\PassOptionsToPackage{dvipsnames,svgnames,x11names}{xcolor}
%
\documentclass[
  11pt,
  a4paper,
  DIV=11,
  numbers=noendperiod]{scrartcl}

\usepackage{amsmath,amssymb}
\usepackage{iftex}
\ifPDFTeX
  \usepackage[T1]{fontenc}
  \usepackage[utf8]{inputenc}
  \usepackage{textcomp} % provide euro and other symbols
\else % if luatex or xetex
  \usepackage{unicode-math}
  \defaultfontfeatures{Scale=MatchLowercase}
  \defaultfontfeatures[\rmfamily]{Ligatures=TeX,Scale=1}
\fi
\usepackage{lmodern}
\ifPDFTeX\else  
    % xetex/luatex font selection
\fi
% Use upquote if available, for straight quotes in verbatim environments
\IfFileExists{upquote.sty}{\usepackage{upquote}}{}
\IfFileExists{microtype.sty}{% use microtype if available
  \usepackage[]{microtype}
  \UseMicrotypeSet[protrusion]{basicmath} % disable protrusion for tt fonts
}{}
\makeatletter
\@ifundefined{KOMAClassName}{% if non-KOMA class
  \IfFileExists{parskip.sty}{%
    \usepackage{parskip}
  }{% else
    \setlength{\parindent}{0pt}
    \setlength{\parskip}{6pt plus 2pt minus 1pt}}
}{% if KOMA class
  \KOMAoptions{parskip=half}}
\makeatother
\usepackage{xcolor}
\usepackage[lmargin=2cm,rmargin=2cm,tmargin=2cm,bmargin=2cm]{geometry}
\setlength{\emergencystretch}{3em} % prevent overfull lines
\setcounter{secnumdepth}{-\maxdimen} % remove section numbering
% Make \paragraph and \subparagraph free-standing
\makeatletter
\ifx\paragraph\undefined\else
  \let\oldparagraph\paragraph
  \renewcommand{\paragraph}{
    \@ifstar
      \xxxParagraphStar
      \xxxParagraphNoStar
  }
  \newcommand{\xxxParagraphStar}[1]{\oldparagraph*{#1}\mbox{}}
  \newcommand{\xxxParagraphNoStar}[1]{\oldparagraph{#1}\mbox{}}
\fi
\ifx\subparagraph\undefined\else
  \let\oldsubparagraph\subparagraph
  \renewcommand{\subparagraph}{
    \@ifstar
      \xxxSubParagraphStar
      \xxxSubParagraphNoStar
  }
  \newcommand{\xxxSubParagraphStar}[1]{\oldsubparagraph*{#1}\mbox{}}
  \newcommand{\xxxSubParagraphNoStar}[1]{\oldsubparagraph{#1}\mbox{}}
\fi
\makeatother

\usepackage{color}
\usepackage{fancyvrb}
\newcommand{\VerbBar}{|}
\newcommand{\VERB}{\Verb[commandchars=\\\{\}]}
\DefineVerbatimEnvironment{Highlighting}{Verbatim}{commandchars=\\\{\}}
% Add ',fontsize=\small' for more characters per line
\usepackage{framed}
\definecolor{shadecolor}{RGB}{241,243,245}
\newenvironment{Shaded}{\begin{snugshade}}{\end{snugshade}}
\newcommand{\AlertTok}[1]{\textcolor[rgb]{0.68,0.00,0.00}{#1}}
\newcommand{\AnnotationTok}[1]{\textcolor[rgb]{0.37,0.37,0.37}{#1}}
\newcommand{\AttributeTok}[1]{\textcolor[rgb]{0.40,0.45,0.13}{#1}}
\newcommand{\BaseNTok}[1]{\textcolor[rgb]{0.68,0.00,0.00}{#1}}
\newcommand{\BuiltInTok}[1]{\textcolor[rgb]{0.00,0.23,0.31}{#1}}
\newcommand{\CharTok}[1]{\textcolor[rgb]{0.13,0.47,0.30}{#1}}
\newcommand{\CommentTok}[1]{\textcolor[rgb]{0.37,0.37,0.37}{#1}}
\newcommand{\CommentVarTok}[1]{\textcolor[rgb]{0.37,0.37,0.37}{\textit{#1}}}
\newcommand{\ConstantTok}[1]{\textcolor[rgb]{0.56,0.35,0.01}{#1}}
\newcommand{\ControlFlowTok}[1]{\textcolor[rgb]{0.00,0.23,0.31}{\textbf{#1}}}
\newcommand{\DataTypeTok}[1]{\textcolor[rgb]{0.68,0.00,0.00}{#1}}
\newcommand{\DecValTok}[1]{\textcolor[rgb]{0.68,0.00,0.00}{#1}}
\newcommand{\DocumentationTok}[1]{\textcolor[rgb]{0.37,0.37,0.37}{\textit{#1}}}
\newcommand{\ErrorTok}[1]{\textcolor[rgb]{0.68,0.00,0.00}{#1}}
\newcommand{\ExtensionTok}[1]{\textcolor[rgb]{0.00,0.23,0.31}{#1}}
\newcommand{\FloatTok}[1]{\textcolor[rgb]{0.68,0.00,0.00}{#1}}
\newcommand{\FunctionTok}[1]{\textcolor[rgb]{0.28,0.35,0.67}{#1}}
\newcommand{\ImportTok}[1]{\textcolor[rgb]{0.00,0.46,0.62}{#1}}
\newcommand{\InformationTok}[1]{\textcolor[rgb]{0.37,0.37,0.37}{#1}}
\newcommand{\KeywordTok}[1]{\textcolor[rgb]{0.00,0.23,0.31}{\textbf{#1}}}
\newcommand{\NormalTok}[1]{\textcolor[rgb]{0.00,0.23,0.31}{#1}}
\newcommand{\OperatorTok}[1]{\textcolor[rgb]{0.37,0.37,0.37}{#1}}
\newcommand{\OtherTok}[1]{\textcolor[rgb]{0.00,0.23,0.31}{#1}}
\newcommand{\PreprocessorTok}[1]{\textcolor[rgb]{0.68,0.00,0.00}{#1}}
\newcommand{\RegionMarkerTok}[1]{\textcolor[rgb]{0.00,0.23,0.31}{#1}}
\newcommand{\SpecialCharTok}[1]{\textcolor[rgb]{0.37,0.37,0.37}{#1}}
\newcommand{\SpecialStringTok}[1]{\textcolor[rgb]{0.13,0.47,0.30}{#1}}
\newcommand{\StringTok}[1]{\textcolor[rgb]{0.13,0.47,0.30}{#1}}
\newcommand{\VariableTok}[1]{\textcolor[rgb]{0.07,0.07,0.07}{#1}}
\newcommand{\VerbatimStringTok}[1]{\textcolor[rgb]{0.13,0.47,0.30}{#1}}
\newcommand{\WarningTok}[1]{\textcolor[rgb]{0.37,0.37,0.37}{\textit{#1}}}

\providecommand{\tightlist}{%
  \setlength{\itemsep}{0pt}\setlength{\parskip}{0pt}}\usepackage{longtable,booktabs,array}
\usepackage{calc} % for calculating minipage widths
% Correct order of tables after \paragraph or \subparagraph
\usepackage{etoolbox}
\makeatletter
\patchcmd\longtable{\par}{\if@noskipsec\mbox{}\fi\par}{}{}
\makeatother
% Allow footnotes in longtable head/foot
\IfFileExists{footnotehyper.sty}{\usepackage{footnotehyper}}{\usepackage{footnote}}
\makesavenoteenv{longtable}
\usepackage{graphicx}
\makeatletter
\def\maxwidth{\ifdim\Gin@nat@width>\linewidth\linewidth\else\Gin@nat@width\fi}
\def\maxheight{\ifdim\Gin@nat@height>\textheight\textheight\else\Gin@nat@height\fi}
\makeatother
% Scale images if necessary, so that they will not overflow the page
% margins by default, and it is still possible to overwrite the defaults
% using explicit options in \includegraphics[width, height, ...]{}
\setkeys{Gin}{width=\maxwidth,height=\maxheight,keepaspectratio}
% Set default figure placement to htbp
\makeatletter
\def\fps@figure{htbp}
\makeatother

\KOMAoption{captions}{tableheading}
\makeatletter
\@ifpackageloaded{caption}{}{\usepackage{caption}}
\AtBeginDocument{%
\ifdefined\contentsname
  \renewcommand*\contentsname{Table of contents}
\else
  \newcommand\contentsname{Table of contents}
\fi
\ifdefined\listfigurename
  \renewcommand*\listfigurename{List of Figures}
\else
  \newcommand\listfigurename{List of Figures}
\fi
\ifdefined\listtablename
  \renewcommand*\listtablename{List of Tables}
\else
  \newcommand\listtablename{List of Tables}
\fi
\ifdefined\figurename
  \renewcommand*\figurename{Figure}
\else
  \newcommand\figurename{Figure}
\fi
\ifdefined\tablename
  \renewcommand*\tablename{Table}
\else
  \newcommand\tablename{Table}
\fi
}
\@ifpackageloaded{float}{}{\usepackage{float}}
\floatstyle{ruled}
\@ifundefined{c@chapter}{\newfloat{codelisting}{h}{lop}}{\newfloat{codelisting}{h}{lop}[chapter]}
\floatname{codelisting}{Listing}
\newcommand*\listoflistings{\listof{codelisting}{List of Listings}}
\makeatother
\makeatletter
\makeatother
\makeatletter
\@ifpackageloaded{caption}{}{\usepackage{caption}}
\@ifpackageloaded{subcaption}{}{\usepackage{subcaption}}
\makeatother

\ifLuaTeX
  \usepackage{selnolig}  % disable illegal ligatures
\fi
\usepackage{bookmark}

\IfFileExists{xurl.sty}{\usepackage{xurl}}{} % add URL line breaks if available
\urlstyle{same} % disable monospaced font for URLs
\hypersetup{
  pdftitle={Türkiye'de Kadınların Sosyoekonomik Durumu Üzerine Veri Analitiği Temelli Bir İnceleme},
  colorlinks=true,
  linkcolor={blue},
  filecolor={Maroon},
  citecolor={Blue},
  urlcolor={Blue},
  pdfcreator={LaTeX via pandoc}}


\title{Türkiye'de Kadınların Sosyoekonomik Durumu Üzerine Veri Analitiği
Temelli Bir İnceleme}
\author{}
\date{}

\begin{document}
\maketitle


\includegraphics[width=2.29167in,height=\textheight]{images/we can do it.gif}

\textbf{Pınar MÜRTEZAOĞLU / Gamze KAZEL BOZKURT}

\subsection{1. Project Overview and
Scope}\label{project-overview-and-scope}

Kadınların sosyoekonomik durumu; işgücüne katılım oranı, gelir seviyesi,
kamusal temsiliyet, eğitim düzeyi ve bölgesel farklılıklar gibi çok
boyutlu faktörlerden etkilenmektedir. Özellikle ataerkil toplumsal
yapının baskın olduğu bölgelerde, kadınlar ekonomik ve sosyal yaşama
katılımda hem yapısal hem de kültürel engellerle karşılaşmaktadır. Bu
durum yalnızca bireysel refahı sınırlamakla kalmamakta, aynı zamanda
ülkenin genel kalkınma potansiyelini de azaltmaktadır. Bu proje,
kadınların Türkiye'deki sosyoekonomik durumunu, eğitim düzeyi, gelir
seviyesi ve toplumsal roller gibi çok boyutlu göstergeler üzerinden veri
temelli bir yaklaşımla analiz etmeyi amaçlamaktadır. Projenin temel
çerçevesi; kadınların eğitim düzeyi ile gelir arasındaki ilişkinin
incelenmesi, işgücüne katılım oranlarının bölgesel farklılıklar
bağlamında analiz edilmesi ve elde edilen veriler ışığında kamu
politikalarına yönelik önerilerin geliştirilmesini içermektedir.

\subsection{2. Data}\label{data}

İstatistiklerle Kadın -- 2024

\subsection{2.1 Data Source}\label{data-source}

Bu çalışmada kullanılan veriler, Türkiye İstatistik Kurumu (TÜİK) veri
tabanından elde edilmiştir. Veriler, TÜİK'in çevrimiçi veri portalından
.xlsx formatında indirilmiş ve analizde kullanılacak şekilde
düzenlenmiştir.

\subsection{2.2 General Information About
Data}\label{general-information-about-data}

Çalışmada; kadınların eğitim düzeyi, işgücüne katılım oranı, bölgesel
düzeyde istihdam oranları, ücret farklılıkları, işgücü içindeki
konumları gibi değişkenler ele alınmış, aynı değişkenler üzerinden
erkeklerle karşılaştırmalı analizler yapılmıştır.

\subsection{2.3 Reason of Choice}\label{reason-of-choice}

Bu çalışmada kullanılan veriler, kamuya açık ve ücretsiz olarak sunulan
TÜİK veri tabanından elde edilmiştir. TÜİK, uluslararası standartlara
uygunluğu ve ulusal düzeyde temsili veri sağlayabilme kapasitesi
nedeniyle tercih edilmiştir.

\subsection{1.Veri}\label{veri}

Bu çalışmada kullanılan İstatistiklerle Kadın -- 2024 verileri, kamuya
açık ve ücretsiz olarak sunulan Türkiye İstatistik Kurumu (TÜİK) veri
tabanından elde edilmiştir. TÜİK, uluslararası standartlara uygunluğu ve
ulusal düzeyde temsili veri sağlayabilme kapasitesi nedeniyle tercih
edilmiştir.

İstatistiklerle Kadın -- 2024 veri setinden; cinsiyet ve eğitim durumuna
göre işgücü durumu, cinsiyet ve İBBS 2. düzeye göre temel işgücü
göstergeleri, cinsiyete göre 25-49 yaş grubunda olup hanehalkında 3
yaşın altında çocuğu olan ve çocuğu olmayan kişilerin istihdam oranı,
~eğitim durumuna göre yıllık ortalama brüt kazanç, cinsiyete göre üst ve
orta düzey yönetici pozisyonlarındaki bireylerin oranı ve meslek grubuna
göre yıllık ortalama brüt kazanç verileri TÜİK'in çevrimiçi veri
portalından .xlsx formatında indirilmiş ve analizde kullanılacak şekilde
ayrı ayrı düzenlenmiştir.

Verinin tamamında bilinmeyen (NA) değer bulunmamaktadır. Yıl, cinsiyet,
işsizlik oranı, istihdam oranı, eğitim düzeyi, kazanç verileri, çocuk
sahibi olma durumu, meslek grupları ve bölgesel farklılıklar sütun
olarak eklenmiş, gözlem değerleri satırlara yazılmıştır.

Cinsiyet verisi kadın ve erkek olarak iki kategoride alınmıştır. Eğitim
düzeyine dair veriler sıralı sayı haline getirilmiştir. Veri
tablolarında eğitim düzeyleri; okuyazar değil, lise altı, lise, mesleki
veya teknik lisesi, yükseköğretim şeklinde beş kategoriye ayrılmıştır.
Analizi kolaylaştırmak için eğitim düzeyleri sırasıyla 1,2,3,4 ve 5
olarak tanımlanmıştır.

İstihdam oranı, işsizlik oranı gibi verilerde yüzdelik olarak verilen
değerlerin yüzde kısımları silinmiş, ondalık sayı(numeric) haline
getirilmiştir.

Bu çalışmada, kadınların eğitim durumu, bölgesel düzeydeki istihdam
oranları, ücret düzeyleri ve işgücü içerisindeki konumları analiz
edilmiş; bu göstergeler doğrultusunda cinsiyete dayalı karşılaştırmalı
değerlendirmeler yapılmıştır.

\subsection{2. İşgücü Piyasasında Genel
Görünüm}\label{iux15fguxfccuxfc-piyasasux131nda-genel-guxf6ruxfcnuxfcm}

\subsection{2.1. Bölgelere Göre İstihdam
Oranı}\label{buxf6lgelere-guxf6re-istihdam-oranux131}

Bölge ve cinsiyete göre istihdam oranlarına ait grafik Şekil-2.1'de
gösterilmektedir. Her bölgede erkek istihdam oranının kadın istihdam
oranından fazla olduğu görülmektedir. 2023 yılında 15 ve daha yukarı
yaştaki kadın nüfusun istihdam oranı \%31,3, erkek nüfusun istihdam
oranı ise \%65,7'dir. İBBS 2.Düzeye göre en yüksek kadın istihdam oranı,
\%38,9 ile TR61 (Antalya, Isparta, Burdur) bölgesinde, en düşük kadın
istihdam oranı ise \%19,8 ile TRC3 (Mardin, Batman, Şırnak, Siirt)
bölgesinde gerçekleşmiştir.

Kadın istihdamının en az olduğu beş bölgeye bakıldığında neredeyse
tamamının Doğu ve Güneydoğu bölgelerindeki illerden oluştuğu dikkat
çekmektedir.

Hemen her bölgede erkeklerin istihdam oranının kadınlara kıyasla anlamlı
şekilde daha yüksek olduğu görülmektedir. Bu durum, toplumsal cinsiyet
temelli istihdam eşitsizliğinin bölgesel düzeyde yaygın bir sorun
olduğunu ortaya koymaktadır.

\begin{Shaded}
\begin{Highlighting}[]
\CommentTok{\# Gerekli paketler}
\CommentTok{\# install.packages("tidyr")}
\CommentTok{\# install.packages("readr")}
\CommentTok{\# install.packages("dplyr")}
\CommentTok{\# install.packages("readxl")}
\CommentTok{\# install.packages("ggplot2")}
\FunctionTok{library}\NormalTok{(readxl)}
\FunctionTok{library}\NormalTok{(ggplot2)}
\FunctionTok{library}\NormalTok{(tidyr)}
\FunctionTok{library}\NormalTok{(dplyr)}
\FunctionTok{library}\NormalTok{(readr)}
\FunctionTok{library}\NormalTok{(patchwork)}
\end{Highlighting}
\end{Shaded}

\subsection{Veriler .RData haline
getirildi.}\label{veriler-.rdata-haline-getirildi.}

\begin{Shaded}
\begin{Highlighting}[]
\NormalTok{bolge\_duzeyi }\OtherTok{\textless{}{-}} \FunctionTok{read\_excel}\NormalTok{(}\StringTok{"bolge\_duzeyi.xlsx"}\NormalTok{)}
\NormalTok{cocuga\_bagli\_istihdam\_orani }\OtherTok{\textless{}{-}} \FunctionTok{read\_excel}\NormalTok{(}\StringTok{"cocuga\_bagli\_istihdam\_orani.xlsx"}\NormalTok{)}
\NormalTok{meslek\_gruplarina\_gore\_kazanc }\OtherTok{\textless{}{-}} \FunctionTok{read\_excel}\NormalTok{(}\StringTok{"meslek\_gruplarina\_gore\_kazanc.xlsx"}\NormalTok{)}
\NormalTok{yillik\_kazanc }\OtherTok{\textless{}{-}} \FunctionTok{read\_excel}\NormalTok{(}\StringTok{"yillik\_kazanc.xlsx"}\NormalTok{)}
\NormalTok{yonetici }\OtherTok{\textless{}{-}} \FunctionTok{read\_excel}\NormalTok{(}\StringTok{"yonetici.xlsx"}\NormalTok{)}
\NormalTok{isgucu\_verisi }\OtherTok{\textless{}{-}} \FunctionTok{read\_excel}\NormalTok{(}\StringTok{"Isgucu\_verisi.xlsx"}\NormalTok{)}

\CommentTok{\# .RData dosyasına kaydet}
\FunctionTok{save}\NormalTok{(}
\NormalTok{  bolge\_duzeyi,}
\NormalTok{  cocuga\_bagli\_istihdam\_orani,}
\NormalTok{  meslek\_gruplarina\_gore\_kazanc,}
\NormalTok{  yillik\_kazanc,}
\NormalTok{  yonetici,}
\NormalTok{  isgucu\_verisi,}
  \AttributeTok{file =} \StringTok{"kadın\_projesi\_verisi.RData"}
\NormalTok{)}
\end{Highlighting}
\end{Shaded}

\begin{Shaded}
\begin{Highlighting}[]
\CommentTok{\# Geçici bir ortamda sadece bolge\_duzeyi verisini yükle}
\NormalTok{temp\_env }\OtherTok{\textless{}{-}} \FunctionTok{new.env}\NormalTok{()}
\FunctionTok{load}\NormalTok{(}\StringTok{"kadın\_projesi\_verisi.RData"}\NormalTok{, }\AttributeTok{envir =}\NormalTok{ temp\_env)}

\CommentTok{\# İlgili veri setini al}
\NormalTok{veri }\OtherTok{\textless{}{-}}\NormalTok{ temp\_env}\SpecialCharTok{$}\NormalTok{bolge\_duzeyi}

\CommentTok{\# Kadın istihdam oranına göre sıralı bölge listesi oluştur}
\NormalTok{sirali\_bolgeler }\OtherTok{\textless{}{-}}\NormalTok{ veri }\SpecialCharTok{\%\textgreater{}\%}
  \FunctionTok{filter}\NormalTok{(cinsiyet }\SpecialCharTok{==} \StringTok{"Kadın"}\NormalTok{) }\SpecialCharTok{\%\textgreater{}\%}
  \FunctionTok{arrange}\NormalTok{(}\FunctionTok{desc}\NormalTok{(istihdam\_orani)) }\SpecialCharTok{\%\textgreater{}\%}
  \FunctionTok{pull}\NormalTok{(bolge)}

\CommentTok{\# Bolgeyi sıralı faktöre çevir (grafikte sıralı görünsün)}
\NormalTok{veri}\SpecialCharTok{$}\NormalTok{bolge }\OtherTok{\textless{}{-}} \FunctionTok{factor}\NormalTok{(veri}\SpecialCharTok{$}\NormalTok{bolge, }\AttributeTok{levels =}\NormalTok{ sirali\_bolgeler)}

\CommentTok{\# Türkiye ortalamaları}
\NormalTok{ortalama\_kadin }\OtherTok{\textless{}{-}} \FloatTok{31.3}
\NormalTok{ortalama\_erkek }\OtherTok{\textless{}{-}} \FloatTok{65.7}

\CommentTok{\# Grafik}
\FunctionTok{ggplot}\NormalTok{(veri, }\FunctionTok{aes}\NormalTok{(}\AttributeTok{x =}\NormalTok{ istihdam\_orani, }\AttributeTok{y =}\NormalTok{ bolge, }\AttributeTok{fill =}\NormalTok{ cinsiyet)) }\SpecialCharTok{+}
  \FunctionTok{geom\_bar}\NormalTok{(}\AttributeTok{stat =} \StringTok{"identity"}\NormalTok{, }\AttributeTok{position =} \StringTok{"dodge"}\NormalTok{) }\SpecialCharTok{+}
  \FunctionTok{geom\_vline}\NormalTok{(}\AttributeTok{xintercept =}\NormalTok{ ortalama\_kadin, }\AttributeTok{color =} \StringTok{"green"}\NormalTok{, }\AttributeTok{linetype =} \StringTok{"dashed"}\NormalTok{, }\AttributeTok{linewidth =} \DecValTok{1}\NormalTok{) }\SpecialCharTok{+}
  \FunctionTok{geom\_vline}\NormalTok{(}\AttributeTok{xintercept =}\NormalTok{ ortalama\_erkek, }\AttributeTok{color =} \StringTok{"steelblue"}\NormalTok{, }\AttributeTok{linetype =} \StringTok{"dashed"}\NormalTok{, }\AttributeTok{linewidth =} \DecValTok{1}\NormalTok{) }\SpecialCharTok{+}
  \FunctionTok{labs}\NormalTok{(}\AttributeTok{title =} \StringTok{"Bolgelere ve Cinsiyete Gore Istihdam Orani"}\NormalTok{,}
       \AttributeTok{x =} \StringTok{"Istihdam Orani (\%)"}\NormalTok{, }\AttributeTok{y =} \StringTok{"Bolge"}\NormalTok{, }\AttributeTok{fill =} \StringTok{"Cinsiyet"}\NormalTok{) }\SpecialCharTok{+}
  \FunctionTok{theme\_minimal}\NormalTok{() }\SpecialCharTok{+}
  \FunctionTok{theme}\NormalTok{(}\AttributeTok{axis.text.y =} \FunctionTok{element\_text}\NormalTok{(}\AttributeTok{size =} \DecValTok{7}\NormalTok{))}
\end{Highlighting}
\end{Shaded}

\includegraphics{project_files/figure-pdf/unnamed-chunk-2-1.pdf}

\subsection{2.2. Eğitim Durumuna Göre Yıllık Kazanç
Durumu}\label{eux11fitim-durumuna-guxf6re-yux131llux131k-kazanuxe7-durumu}

Eğitim düzeyine göre kadın ve erkek için yıllık kazanç miktarları
Şekil-2.2'deki grafikte gösterilmektedir. 2023 yılı verilerine göre
kazanç düzeylerinin hem erkeklerde hem de kadınlarda eğitim durumu ile
birlikte yükseldiği görülmektedir. Ancak eğitim seviyesi arttıkça
cinsiyetler arası kazanç farkı belirginleşmektedir. Bu durum, özellikle
yükseköğretim mezunları arasında ciddi bir gelir eşitsizliğine işaret
etmektedir. Eğitim durumuna göre en yüksek yıllık ortalama brüt kazancı
yükseköğretim eğitim düzeyine sahip olanlar elde etmiş olup, bu eğitim
düzeyinde yıllık ortalama brüt kazanç erkeklerde 431 bin 364 TL,
kadınlarda ise 354 bin 149 TL olmuştur. Tüm seviyelerde erkek ve kadın
aynı eğitim düzeyinde olmasına rağmen erkeklerin yıllık ortalama brüt
kazancının kadınlardan fazla olduğu görülmektedir.

\begin{Shaded}
\begin{Highlighting}[]
\FunctionTok{library}\NormalTok{(readxl)}
\FunctionTok{library}\NormalTok{(ggplot2)}
\FunctionTok{library}\NormalTok{(tidyr)}
\FunctionTok{library}\NormalTok{(dplyr)}
\FunctionTok{library}\NormalTok{(readr)}

\CommentTok{\# Geçici bir ortam oluştur ve .RData dosyasını yükle}
\NormalTok{temp\_env }\OtherTok{\textless{}{-}} \FunctionTok{new.env}\NormalTok{()}
\FunctionTok{load}\NormalTok{(}\StringTok{"kadın\_projesi\_verisi.RData"}\NormalTok{, }\AttributeTok{envir =}\NormalTok{ temp\_env)}

\CommentTok{\# Sadece yillik\_kazanc verisini al}
\NormalTok{veri }\OtherTok{\textless{}{-}}\NormalTok{ temp\_env}\SpecialCharTok{$}\NormalTok{yillik\_kazanc}

\CommentTok{\# Sütun adlarını ASCII karakterlerine dönüştür}
\FunctionTok{names}\NormalTok{(veri) }\OtherTok{\textless{}{-}} \FunctionTok{iconv}\NormalTok{(}\FunctionTok{names}\NormalTok{(veri), }\AttributeTok{from =} \StringTok{""}\NormalTok{, }\AttributeTok{to =} \StringTok{"ASCII//TRANSLIT"}\NormalTok{)}

\CommentTok{\# Grafik oluştur}
\FunctionTok{ggplot}\NormalTok{(veri, }\FunctionTok{aes}\NormalTok{(}\AttributeTok{x =} \FunctionTok{reorder}\NormalTok{(egitim\_duzeyi, yillik\_ort\_brut\_kazanc), }
                 \AttributeTok{y =}\NormalTok{ yillik\_ort\_brut\_kazanc, }\AttributeTok{fill =}\NormalTok{ cinsiyet)) }\SpecialCharTok{+}
  \FunctionTok{geom\_bar}\NormalTok{(}\AttributeTok{stat =} \StringTok{"identity"}\NormalTok{, }\AttributeTok{position =} \StringTok{"dodge"}\NormalTok{) }\SpecialCharTok{+}
  \FunctionTok{geom\_text}\NormalTok{(}\FunctionTok{aes}\NormalTok{(}\AttributeTok{label =} \FunctionTok{paste0}\NormalTok{(}\FunctionTok{format}\NormalTok{(yillik\_ort\_brut\_kazanc, }\AttributeTok{big.mark =} \StringTok{"."}\NormalTok{, }\AttributeTok{decimal.mark =} \StringTok{","}\NormalTok{), }\StringTok{",00"}\NormalTok{)), }
            \AttributeTok{position =} \FunctionTok{position\_dodge}\NormalTok{(}\AttributeTok{width =} \FloatTok{0.9}\NormalTok{), }\AttributeTok{vjust =} \SpecialCharTok{{-}}\FloatTok{0.3}\NormalTok{, }\AttributeTok{size =} \DecValTok{3}\NormalTok{) }\SpecialCharTok{+} 
  \FunctionTok{scale\_y\_continuous}\NormalTok{(}\AttributeTok{labels =}\NormalTok{ scales}\SpecialCharTok{::}\FunctionTok{comma\_format}\NormalTok{(}\AttributeTok{big.mark =} \StringTok{"."}\NormalTok{, }\AttributeTok{decimal.mark =} \StringTok{","}\NormalTok{)) }\SpecialCharTok{+}
  \FunctionTok{labs}\NormalTok{(}\AttributeTok{title =} \StringTok{"Egitim Duzeyine Gore Yillik Ortalama Brut Kazanc"}\NormalTok{,}
       \AttributeTok{x =} \StringTok{"Egitim Duzeyi"}\NormalTok{, }\AttributeTok{y =} \StringTok{"Yillik Kazanc (TL)"}\NormalTok{, }\AttributeTok{fill =} \StringTok{"Cinsiyet"}\NormalTok{) }\SpecialCharTok{+}
  \FunctionTok{theme\_minimal}\NormalTok{()}
\end{Highlighting}
\end{Shaded}

\includegraphics{project_files/figure-pdf/unnamed-chunk-3-1.pdf}

\subsection{2.3. Meslek Gruplarına Göre Kazanç
Durumu}\label{meslek-gruplarux131na-guxf6re-kazanuxe7-durumu}

Meslek gruplarına göre kadın ve erkek için yıllık ortalama kazanç
verileri Şekil-2.3'te gösterilmektedir. 2023 yılı verilerine göre,
kadınlarda en yüksek yıllık ortalama brüt kazancı 530.663,00 TL ile
yöneticiler meslek grubunda çalışanlar elde etmiştir. Yönetici
pozisyonunda çalışan erkek ve kadınlar arasında yıllık ortalama
kazançlarında farklılığın düşük olduğu dikkat çekmektedir. En düşük
yıllık ortalama brüt kazanç ise 185.860,00 TL ile nitelikli tarım,
ormancılık ve su ürünleri çalışanları grubunda gerçekleşmiştir. Yalnızca
``Hizmet ve servis elemanları'' meslek grubunda çalışan kadınlar aynı
meslek grubunda çalışan erkeklere göre daha fazla yıllık ortalama kazanç
elde etmiş, diğer tüm meslek gruplarında erkekler kadınlardan daha fazla
yıllık ortalama kazanç sağlamıştır.

\begin{Shaded}
\begin{Highlighting}[]
\FunctionTok{library}\NormalTok{(readxl)}
\FunctionTok{library}\NormalTok{(ggplot2)}
\FunctionTok{library}\NormalTok{(tidyr)}
\FunctionTok{library}\NormalTok{(dplyr)}
\FunctionTok{library}\NormalTok{(readr)}
\FunctionTok{library}\NormalTok{(scales)}


\CommentTok{\# Geçici bir ortam oluştur ve RData dosyasını yükle}
\NormalTok{temp\_env }\OtherTok{\textless{}{-}} \FunctionTok{new.env}\NormalTok{()}
\FunctionTok{load}\NormalTok{(}\StringTok{"kadın\_projesi\_verisi.RData"}\NormalTok{, }\AttributeTok{envir =}\NormalTok{ temp\_env)}

\CommentTok{\# İlgili veri setini al}
\NormalTok{veri }\OtherTok{\textless{}{-}}\NormalTok{ temp\_env}\SpecialCharTok{$}\NormalTok{meslek\_gruplarina\_gore\_kazanc}


\CommentTok{\# Veriyi meslek ve cinsiyete göre sıralayın}
\NormalTok{veri }\OtherTok{\textless{}{-}}\NormalTok{ veri }\SpecialCharTok{\%\textgreater{}\%}
  \FunctionTok{group\_by}\NormalTok{(meslek, cinsiyet) }\SpecialCharTok{\%\textgreater{}\%}
  \FunctionTok{summarize}\NormalTok{(}\AttributeTok{yillik\_ort\_kazanc =} \FunctionTok{mean}\NormalTok{(yillik\_ort\_kazanc, }\AttributeTok{na.rm =} \ConstantTok{TRUE}\NormalTok{, }\AttributeTok{.groups =} \StringTok{"drop"}\NormalTok{))}

\CommentTok{\# Kadınlara göre meslek sıralaması}
\NormalTok{sirali\_meslekler }\OtherTok{\textless{}{-}}\NormalTok{ veri }\SpecialCharTok{\%\textgreater{}\%}
  \FunctionTok{filter}\NormalTok{(cinsiyet }\SpecialCharTok{==} \StringTok{"Kadın"}\NormalTok{) }\SpecialCharTok{\%\textgreater{}\%}
  \FunctionTok{arrange}\NormalTok{(yillik\_ort\_kazanc) }\SpecialCharTok{\%\textgreater{}\%}
  \FunctionTok{pull}\NormalTok{(meslek)}

\CommentTok{\# Sıralı faktör olarak ayarla}
\NormalTok{veri}\SpecialCharTok{$}\NormalTok{meslek }\OtherTok{\textless{}{-}} \FunctionTok{factor}\NormalTok{(veri}\SpecialCharTok{$}\NormalTok{meslek, }\AttributeTok{levels =}\NormalTok{ sirali\_meslekler)}

\CommentTok{\# Grafik oluşturma}
\FunctionTok{ggplot}\NormalTok{(veri, }\FunctionTok{aes}\NormalTok{(}\AttributeTok{x =}\NormalTok{ meslek, }\AttributeTok{y =}\NormalTok{ yillik\_ort\_kazanc, }\AttributeTok{fill =}\NormalTok{ cinsiyet)) }\SpecialCharTok{+}
  \FunctionTok{geom\_bar}\NormalTok{(}\AttributeTok{stat =} \StringTok{"identity"}\NormalTok{, }\AttributeTok{position =} \StringTok{"dodge"}\NormalTok{) }\SpecialCharTok{+}
  \FunctionTok{geom\_text}\NormalTok{(}\FunctionTok{aes}\NormalTok{(}\AttributeTok{label =} \FunctionTok{format}\NormalTok{(yillik\_ort\_kazanc, }\AttributeTok{big.mark =} \StringTok{"."}\NormalTok{, }\AttributeTok{decimal.mark =} \StringTok{","}\NormalTok{, }\AttributeTok{nsmall =} \DecValTok{2}\NormalTok{)),}
            \AttributeTok{position =} \FunctionTok{position\_dodge}\NormalTok{(}\AttributeTok{width =} \FloatTok{0.9}\NormalTok{), }
            \AttributeTok{hjust =} \SpecialCharTok{{-}}\FloatTok{0.1}\NormalTok{, }\AttributeTok{size =} \FloatTok{2.5}\NormalTok{) }\SpecialCharTok{+}  \CommentTok{\# ← etiketler çubuğun sağında}
  \FunctionTok{coord\_flip}\NormalTok{() }\SpecialCharTok{+}
  \FunctionTok{scale\_fill\_manual}\NormalTok{(}\AttributeTok{values =} \FunctionTok{c}\NormalTok{(}\StringTok{"Kadın"} \OtherTok{=} \StringTok{"yellow"}\NormalTok{, }\StringTok{"Erkek"} \OtherTok{=} \StringTok{"lightgreen"}\NormalTok{)) }\SpecialCharTok{+}
  \FunctionTok{scale\_y\_continuous}\NormalTok{(}
    \AttributeTok{breaks =} \FunctionTok{c}\NormalTok{(}\DecValTok{0}\NormalTok{, }\DecValTok{100000}\NormalTok{, }\DecValTok{300000}\NormalTok{, }\DecValTok{500000}\NormalTok{),}
    \AttributeTok{labels =} \FunctionTok{label\_number}\NormalTok{(}\AttributeTok{big.mark =} \StringTok{"."}\NormalTok{, }\AttributeTok{decimal.mark =} \StringTok{","}\NormalTok{, }\AttributeTok{accuracy =} \DecValTok{1}\NormalTok{),}
    \AttributeTok{expand =} \FunctionTok{expansion}\NormalTok{(}\AttributeTok{mult =} \FunctionTok{c}\NormalTok{(}\DecValTok{0}\NormalTok{, }\FloatTok{0.25}\NormalTok{))  }\CommentTok{\# ← boşluk bırak, çubukların sonunda etiketler için alan yarat}
\NormalTok{  ) }\SpecialCharTok{+}
  \FunctionTok{labs}\NormalTok{(}\AttributeTok{title =} \StringTok{"Ortalama Kazanc"}\NormalTok{,}
       \AttributeTok{x =} \StringTok{"Meslek Grubu"}\NormalTok{, }\AttributeTok{y =} \StringTok{"Yillik Ortalama Kazanc (TL)"}\NormalTok{, }\AttributeTok{fill =} \StringTok{"Cinsiyet"}\NormalTok{) }\SpecialCharTok{+}
  \FunctionTok{theme\_minimal}\NormalTok{() }\SpecialCharTok{+}
  \FunctionTok{theme}\NormalTok{(}\AttributeTok{axis.text.y =} \FunctionTok{element\_text}\NormalTok{(}\AttributeTok{margin =} \FunctionTok{margin}\NormalTok{(}\AttributeTok{r =} \DecValTok{10}\NormalTok{)))}
\end{Highlighting}
\end{Shaded}

\includegraphics{project_files/figure-pdf/unnamed-chunk-4-1.pdf}

\subsection{2.4. Üst ve Orta Düzey Yönetici Pozisyonundaki
Görünüm}\label{uxfcst-ve-orta-duxfczey-yuxf6netici-pozisyonundaki-guxf6ruxfcnuxfcm}

Yıllara göre kadın ve erkek için üst ve orta düzey yönetici olma oranı
Şekil-2.4'teki grafikte gösterilmektedir. Üst ve orta düzey yönetici
pozisyonundaki kadın oranı 2012 yılında \%14.4 iken 2023 yılında \%20.6
olmuştur. Bu artış, kadınların liderlik pozisyonlarında görünürlüğünün
zamanla arttığını göstermektedir. Ancak artış hızı yavaş ve sınırlı
kalmıştır. Bu oran erkeklerde 2012 yılında \%85.6 iken 2023 yılında
\%79.4 olarak gerçekleşmiştir. Bu düşüş, kadınların yönetime daha fazla
dâhil olmaya başladığını gösterse de, erkekler hâlâ yönetici
pozisyonlarının büyük çoğunluğunu elinde bulundurmaktadır. 2012-2023
yıllarındaki orta ve üst düzey yönetici oranları incelendiğinde tüm
yıllarda kadınların yönetici pozisyonunda erkeklerle kıyaslandığında çok
daha az yer bulabildiği görülmektedir.

\begin{Shaded}
\begin{Highlighting}[]
\FunctionTok{library}\NormalTok{(readxl)}
\FunctionTok{library}\NormalTok{(ggplot2)}
\FunctionTok{library}\NormalTok{(tidyr)}
\FunctionTok{library}\NormalTok{(dplyr)}
\FunctionTok{library}\NormalTok{(readr)}

\CommentTok{\# Geçici ortam oluştur ve .RData dosyasını yükle}
\NormalTok{temp\_env }\OtherTok{\textless{}{-}} \FunctionTok{new.env}\NormalTok{()}
\FunctionTok{load}\NormalTok{(}\StringTok{"kadın\_projesi\_verisi.RData"}\NormalTok{, }\AttributeTok{envir =}\NormalTok{ temp\_env)}

\CommentTok{\# Sadece \textquotesingle{}yonetici\textquotesingle{} verisini al}
\NormalTok{veri }\OtherTok{\textless{}{-}}\NormalTok{ temp\_env}\SpecialCharTok{$}\NormalTok{yonetici}

\CommentTok{\# Sütun adlarını ASCII formatına çevir (Türkçe karakter hatası önlemi)}
\FunctionTok{names}\NormalTok{(veri) }\OtherTok{\textless{}{-}} \FunctionTok{iconv}\NormalTok{(}\FunctionTok{names}\NormalTok{(veri), }\AttributeTok{from =} \StringTok{""}\NormalTok{, }\AttributeTok{to =} \StringTok{"ASCII//TRANSLIT"}\NormalTok{)}

\CommentTok{\# Veriyi düzenle}
\NormalTok{veri }\OtherTok{\textless{}{-}}\NormalTok{ veri }\SpecialCharTok{\%\textgreater{}\%}
  \FunctionTok{mutate}\NormalTok{(}
    \AttributeTok{cinsiyet =} \FunctionTok{as.character}\NormalTok{(cinsiyet),}
    \AttributeTok{orta\_ust\_yonetici\_orani =} \FunctionTok{as.numeric}\NormalTok{(orta\_ust\_yonetici\_orani)}
\NormalTok{  ) }\SpecialCharTok{\%\textgreater{}\%}
  \FunctionTok{filter}\NormalTok{(cinsiyet }\SpecialCharTok{\%in\%} \FunctionTok{c}\NormalTok{(}\StringTok{"Kadın"}\NormalTok{, }\StringTok{"Erkek"}\NormalTok{)) }\SpecialCharTok{\%\textgreater{}\%}
  \FunctionTok{mutate}\NormalTok{(}\AttributeTok{cinsiyet =} \FunctionTok{factor}\NormalTok{(cinsiyet, }\AttributeTok{levels =} \FunctionTok{c}\NormalTok{(}\StringTok{"Kadın"}\NormalTok{, }\StringTok{"Erkek"}\NormalTok{)))}

\CommentTok{\# Grafik oluştur}
\FunctionTok{ggplot}\NormalTok{(veri, }\FunctionTok{aes}\NormalTok{(}\AttributeTok{x =} \FunctionTok{as.factor}\NormalTok{(yil), }\AttributeTok{y =}\NormalTok{ orta\_ust\_yonetici\_orani, }\AttributeTok{color =}\NormalTok{ cinsiyet, }\AttributeTok{group =}\NormalTok{ cinsiyet)) }\SpecialCharTok{+}
  \FunctionTok{geom\_line}\NormalTok{(}\AttributeTok{linewidth =} \DecValTok{1}\NormalTok{) }\SpecialCharTok{+}
  \FunctionTok{geom\_point}\NormalTok{(}\AttributeTok{size =} \DecValTok{2}\NormalTok{) }\SpecialCharTok{+}
  \FunctionTok{geom\_text}\NormalTok{(}
    \FunctionTok{aes}\NormalTok{(}\AttributeTok{label =} \FunctionTok{sprintf}\NormalTok{(}\StringTok{"\%.1f"}\NormalTok{, orta\_ust\_yonetici\_orani)),}
    \AttributeTok{vjust =} \SpecialCharTok{{-}}\FloatTok{0.5}\NormalTok{,}
    \AttributeTok{size =} \DecValTok{3}\NormalTok{,}
    \AttributeTok{show.legend =} \ConstantTok{FALSE}
\NormalTok{  ) }\SpecialCharTok{+}
  \FunctionTok{labs}\NormalTok{(}\AttributeTok{title =} \StringTok{"Yillara Gore Orta ve Ust Duzey Yonetici Olma Orani"}\NormalTok{,}
       \AttributeTok{x =} \StringTok{"Yil"}\NormalTok{,}
       \AttributeTok{y =} \StringTok{"Orta ve Ust Duzey Yonetici Orani (\%)"}\NormalTok{,}
       \AttributeTok{color =} \StringTok{"Cinsiyet"}\NormalTok{) }\SpecialCharTok{+}
  \FunctionTok{scale\_color\_manual}\NormalTok{(}\AttributeTok{values =} \FunctionTok{c}\NormalTok{(}\StringTok{"Kadın"} \OtherTok{=} \StringTok{"purple"}\NormalTok{, }\StringTok{"Erkek"} \OtherTok{=} \StringTok{"orange"}\NormalTok{)) }\SpecialCharTok{+}
  \FunctionTok{theme\_minimal}\NormalTok{()}
\end{Highlighting}
\end{Shaded}

\includegraphics{project_files/figure-pdf/unnamed-chunk-5-1.pdf}

\subsection{2.5. Çocuğa Bağlı İstihdam
Oranı}\label{uxe7ocuux11fa-baux11flux131-istihdam-oranux131}

Üç yaş altı çocuk sahibi olma durumuna göre kadın ve erkek için yıllara
göre istihdam oranları Şekil-2.5'te gösterilmektedir. 2023 yılında
hanesinde 3 yaşın altında çocuğu olan 25-49 yaş grubundaki kadınların
istihdam oranının \%27,1, erkeklerin istihdam oranının ise \%90,6 olduğu
görülmüştür. 2023 yılında çocuk sahibi olmayan kadınların istihdam oranı
\%58 iken, erkekler için bu oran \%79.3'tür.

2014-2023 yılları için veriler incelendiğinde çocuk sahibi olmayan
kadınların istihdam oranının, 3 yaş altı çocuk sahibi olan kadınlara
kıyasla tüm yıllar için neredeyse iki katı olduğu görülmektedir. Bu
durum erkeklerde ters etki yaratmaktadır. 3 yaş altı çocuk sahibi olan
erkeklerin istihdam oranı, çocuk sahibi olmayanlara göre daha fazladır.
Bu grafik, bakım yükümlülüklerinin cinsiyetler arasında eşit
dağılmadığını açıkça göstermektedir. Erkeklerin çocuk sahibi olduktan
sonra istihdam oranları büyük ölçüde korunurken, kadınların istihdam
oranı keskin şekilde düşmektedir.

\begin{Shaded}
\begin{Highlighting}[]
\FunctionTok{library}\NormalTok{(readxl)}
\FunctionTok{library}\NormalTok{(ggplot2)}
\FunctionTok{library}\NormalTok{(tidyr)}
\FunctionTok{library}\NormalTok{(dplyr)}
\FunctionTok{library}\NormalTok{(readr)}

\CommentTok{\# Geçici bir ortam oluştur ve .RData dosyasını yükle}
\NormalTok{temp\_env }\OtherTok{\textless{}{-}} \FunctionTok{new.env}\NormalTok{()}
\FunctionTok{load}\NormalTok{(}\StringTok{"kadın\_projesi\_verisi.RData"}\NormalTok{, }\AttributeTok{envir =}\NormalTok{ temp\_env)}

\CommentTok{\# Veriyi çağır}
\NormalTok{veri }\OtherTok{\textless{}{-}}\NormalTok{ temp\_env}\SpecialCharTok{$}\NormalTok{cocuga\_bagli\_istihdam\_orani}

\CommentTok{\# Uzun formata dönüştür (pivot\_longer)}
\NormalTok{veri\_long }\OtherTok{\textless{}{-}}\NormalTok{ veri }\SpecialCharTok{\%\textgreater{}\%}
  \FunctionTok{pivot\_longer}\NormalTok{(}
    \AttributeTok{cols =} \FunctionTok{c}\NormalTok{(}\StringTok{"3yasalti\_cocuk\_olan\_istihdam\_orani"}\NormalTok{, }\StringTok{"cocuk\_olmayan\_istihdam\_orani"}\NormalTok{), }
    \AttributeTok{names\_to =} \StringTok{"cocuk\_durumu"}\NormalTok{, }
    \AttributeTok{values\_to =} \StringTok{"istihdam\_orani"}
\NormalTok{  ) }\SpecialCharTok{\%\textgreater{}\%}
  \FunctionTok{mutate}\NormalTok{(}\AttributeTok{cocuk\_durumu =} \FunctionTok{recode}\NormalTok{(cocuk\_durumu, }
    \StringTok{"3yasalti\_cocuk\_olan\_istihdam\_orani"} \OtherTok{=} \StringTok{"3 Yaş Altı Çocuk Sahibi"}\NormalTok{,}
    \StringTok{"cocuk\_olmayan\_istihdam\_orani"} \OtherTok{=} \StringTok{"Çocuk Sahibi Değil"}
\NormalTok{  ))}

\CommentTok{\# Grafik oluştur}
\FunctionTok{ggplot}\NormalTok{(veri\_long, }\FunctionTok{aes}\NormalTok{(}\AttributeTok{x =} \FunctionTok{as.factor}\NormalTok{(yil), }\AttributeTok{y =}\NormalTok{ istihdam\_orani, }\AttributeTok{color =}\NormalTok{ cinsiyet, }
                      \AttributeTok{linetype =}\NormalTok{ cocuk\_durumu, }\AttributeTok{group =} \FunctionTok{interaction}\NormalTok{(cinsiyet, cocuk\_durumu))) }\SpecialCharTok{+}
  \FunctionTok{geom\_line}\NormalTok{(}\AttributeTok{linewidth =} \FloatTok{1.2}\NormalTok{) }\SpecialCharTok{+}
  \FunctionTok{geom\_point}\NormalTok{(}\AttributeTok{size =} \DecValTok{3}\NormalTok{) }\SpecialCharTok{+}
  \FunctionTok{geom\_text}\NormalTok{(}\FunctionTok{aes}\NormalTok{(}\AttributeTok{label =} \FunctionTok{round}\NormalTok{(istihdam\_orani, }\DecValTok{1}\NormalTok{)), }
            \AttributeTok{vjust =} \SpecialCharTok{{-}}\FloatTok{0.9}\NormalTok{, }\AttributeTok{size =} \DecValTok{3}\NormalTok{, }\AttributeTok{show.legend =} \ConstantTok{FALSE}\NormalTok{) }\SpecialCharTok{+}
  \FunctionTok{labs}\NormalTok{(}\AttributeTok{title =} \StringTok{"Yillara Gore 3 Yas Alti Cocuk Sahibi Olma Durumuna Gore Istihdam Orani"}\NormalTok{,}
       \AttributeTok{x =} \StringTok{"Yil"}\NormalTok{,}
       \AttributeTok{y =} \StringTok{"Istihdam Orani (\%)"}\NormalTok{,}
       \AttributeTok{color =} \StringTok{"Cinsiyet"}\NormalTok{,}
       \AttributeTok{linetype =} \StringTok{"Cocuk Durumu"}\NormalTok{) }\SpecialCharTok{+}
  \FunctionTok{scale\_x\_discrete}\NormalTok{(}\AttributeTok{breaks =} \FunctionTok{unique}\NormalTok{(veri\_long}\SpecialCharTok{$}\NormalTok{yil)) }\SpecialCharTok{+}
  \FunctionTok{scale\_y\_continuous}\NormalTok{(}\AttributeTok{breaks =} \FunctionTok{seq}\NormalTok{(}\DecValTok{0}\NormalTok{, }\DecValTok{90}\NormalTok{, }\AttributeTok{by =} \DecValTok{10}\NormalTok{)) }\SpecialCharTok{+}
  \FunctionTok{theme\_minimal}\NormalTok{() }\SpecialCharTok{+}
  \FunctionTok{scale\_color\_manual}\NormalTok{(}\AttributeTok{values =} \FunctionTok{c}\NormalTok{(}\StringTok{"royalblue"}\NormalTok{, }\StringTok{"hotpink"}\NormalTok{))}
\end{Highlighting}
\end{Shaded}

\includegraphics{project_files/figure-pdf/unnamed-chunk-6-1.pdf}

\subsection{3. Analiz}\label{analiz}

Çalışmanın temel amacı, Türkiye'de işgücü piyasasında eğitim seviyesine
göre cinsiyet farklılıklarını belirlemektir. Bu kapsamda, işgücü
verileri kullanılarak iki temel model oluşturulmuştur: İstihdam Oranı
Modeli~ ve İşsizlik Oranı Modeli

\subsection{3.1. Modelleme
Yaklaşımı}\label{modelleme-yaklaux15fux131mux131}

Analizde, cinsiyet ve eğitim düzeyi arasındaki etkileşimi gözlemlemek
için doğrusal regresyon modelleri kullanılmıştır.

\subsection{3.1.1. İstihdam Oranı
Modeli}\label{istihdam-oranux131-modeli}

İstihdam oranı, işgücünün hangi oranda ekonomik faaliyetlere
katılabildiğini göstermesi bakımından önemlidir. İstihdam edilen
kişilerin kurumsal olmayan çalışma çağındaki nüfusa oranı, istihdam
oranını ifade etmektedir. Bu modelde, istihdam oranı bağımlı değişken
olarak alınmış, eğitim düzeyi ve cinsiyet bağımsız değişken olarak
modele dahil edilmiştir. Model, eğitim düzeyi ve cinsiyet arasındaki
etkileşimi de göz önünde bulunduracak şekilde formüle edilmiştir:

İstihdam Oranı = β0 + β1 × Eğitim Düzeyi + β2 × Cinsiyet + β3 × (Eğitim
Düzeyi × Cinsiyet) + ϵ

\subsection{3.1.2. İşsizlik Oranı
Modeli}\label{iux15fsizlik-oranux131-modeli}

İşsizlik oranı, bir ekonomide işi olmayıp iş arayanların işgücüne
oranıdır. İşsizlik oranı ekonominin performansını yansıtması bakımından
oldukça önemlidir. Bu modelde işsizlik oranı bağımlı değişken olarak
kullanılmış ve benzer şekilde eğitim düzeyi ile cinsiyet arasındaki
etkileşim dikkate alınarak analiz edilmiştir:

İşsizlik Oranı = β0 + β1 × Eğitim Düzeyi + β2 × Cinsiyet + β3 × (Eğitim
Düzeyi × Cinsiyet) + ϵ

\begin{Shaded}
\begin{Highlighting}[]
\CommentTok{\# Geçici ortam oluştur ve RData dosyasını yükle}
\NormalTok{temp\_env }\OtherTok{\textless{}{-}} \FunctionTok{new.env}\NormalTok{()}
\FunctionTok{load}\NormalTok{(}\StringTok{"kadın\_projesi\_verisi.RData"}\NormalTok{, }\AttributeTok{envir =}\NormalTok{ temp\_env)}

\CommentTok{\# Veriyi al}
\NormalTok{veri }\OtherTok{\textless{}{-}}\NormalTok{ temp\_env}\SpecialCharTok{$}\NormalTok{isgucu\_verisi}

\CommentTok{\# Eksik verileri grup ortalaması ile doldur}
\NormalTok{veri }\OtherTok{\textless{}{-}}\NormalTok{ veri }\SpecialCharTok{\%\textgreater{}\%}
  \FunctionTok{group\_by}\NormalTok{(cinsiyet, egitim\_duzeyi) }\SpecialCharTok{\%\textgreater{}\%}
  \FunctionTok{mutate}\NormalTok{(}
    \AttributeTok{issizlik\_orani =} \FunctionTok{ifelse}\NormalTok{(}\FunctionTok{is.na}\NormalTok{(issizlik\_orani), }\FunctionTok{mean}\NormalTok{(issizlik\_orani, }\AttributeTok{na.rm =} \ConstantTok{TRUE}\NormalTok{), issizlik\_orani),}
    \AttributeTok{istihdam\_orani =} \FunctionTok{ifelse}\NormalTok{(}\FunctionTok{is.na}\NormalTok{(istihdam\_orani), }\FunctionTok{mean}\NormalTok{(istihdam\_orani, }\AttributeTok{na.rm =} \ConstantTok{TRUE}\NormalTok{), istihdam\_orani)}
\NormalTok{  ) }\SpecialCharTok{\%\textgreater{}\%}
  \FunctionTok{ungroup}\NormalTok{()}

\CommentTok{\# Eğitim düzeyine sıralama ekle}
\NormalTok{veri }\OtherTok{\textless{}{-}}\NormalTok{ veri }\SpecialCharTok{\%\textgreater{}\%}
  \FunctionTok{mutate}\NormalTok{(}\AttributeTok{egitim\_sira =} \FunctionTok{case\_when}\NormalTok{(}
\NormalTok{    egitim\_duzeyi }\SpecialCharTok{==} \StringTok{"Okuma yazma bilmeyen"} \SpecialCharTok{\textasciitilde{}} \DecValTok{1}\NormalTok{,}
\NormalTok{    egitim\_duzeyi }\SpecialCharTok{==} \StringTok{"Ilkogretim"} \SpecialCharTok{\textasciitilde{}} \DecValTok{2}\NormalTok{,}
\NormalTok{    egitim\_duzeyi }\SpecialCharTok{==} \StringTok{"Genel lise"} \SpecialCharTok{\textasciitilde{}} \DecValTok{3}\NormalTok{,}
\NormalTok{    egitim\_duzeyi }\SpecialCharTok{==} \StringTok{"Lise dengi mesleki okul"} \SpecialCharTok{\textasciitilde{}} \DecValTok{4}\NormalTok{,}
\NormalTok{    egitim\_duzeyi }\SpecialCharTok{==} \StringTok{"Yuksekogretim"} \SpecialCharTok{\textasciitilde{}} \DecValTok{5}\NormalTok{,}
    \ConstantTok{TRUE} \SpecialCharTok{\textasciitilde{}} \ConstantTok{NA\_real\_}
\NormalTok{  ))}

\CommentTok{\# Cinsiyeti faktöre çevir}
\NormalTok{veri}\SpecialCharTok{$}\NormalTok{cinsiyet }\OtherTok{\textless{}{-}} \FunctionTok{factor}\NormalTok{(veri}\SpecialCharTok{$}\NormalTok{cinsiyet)}

\CommentTok{\# Regresyon modelleri}
\NormalTok{model\_istihdam }\OtherTok{\textless{}{-}} \FunctionTok{lm}\NormalTok{(istihdam\_orani }\SpecialCharTok{\textasciitilde{}}\NormalTok{ egitim\_sira }\SpecialCharTok{*}\NormalTok{ cinsiyet, }\AttributeTok{data =}\NormalTok{ veri)}
\NormalTok{model\_issizlik }\OtherTok{\textless{}{-}} \FunctionTok{lm}\NormalTok{(issizlik\_orani }\SpecialCharTok{\textasciitilde{}}\NormalTok{ egitim\_sira }\SpecialCharTok{*}\NormalTok{ cinsiyet, }\AttributeTok{data =}\NormalTok{ veri)}

\CommentTok{\# Grafik tema ve renk ayarları}
\NormalTok{my\_colors }\OtherTok{\textless{}{-}} \FunctionTok{c}\NormalTok{(}\StringTok{"Kadın"} \OtherTok{=} \StringTok{"darkred"}\NormalTok{, }\StringTok{"Erkek"} \OtherTok{=} \StringTok{"steelblue"}\NormalTok{)}
\NormalTok{my\_theme }\OtherTok{\textless{}{-}} \FunctionTok{theme\_minimal}\NormalTok{(}\AttributeTok{base\_size =} \DecValTok{12}\NormalTok{)}

\CommentTok{\# Grafik 1: İstihdam Oranı}
\NormalTok{g1 }\OtherTok{\textless{}{-}} \FunctionTok{ggplot}\NormalTok{(veri, }\FunctionTok{aes}\NormalTok{(}\AttributeTok{x =}\NormalTok{ egitim\_sira, }\AttributeTok{y =}\NormalTok{ istihdam\_orani, }\AttributeTok{color =}\NormalTok{ cinsiyet)) }\SpecialCharTok{+}
  \FunctionTok{geom\_point}\NormalTok{(}\AttributeTok{size =} \DecValTok{3}\NormalTok{) }\SpecialCharTok{+}
  \FunctionTok{geom\_smooth}\NormalTok{(}\AttributeTok{method =} \StringTok{"lm"}\NormalTok{, }\AttributeTok{se =} \ConstantTok{FALSE}\NormalTok{, }\AttributeTok{linewidth =} \FloatTok{1.2}\NormalTok{) }\SpecialCharTok{+}
  \FunctionTok{scale\_color\_manual}\NormalTok{(}\AttributeTok{values =}\NormalTok{ my\_colors) }\SpecialCharTok{+}
  \FunctionTok{scale\_x\_continuous}\NormalTok{(}
    \AttributeTok{breaks =} \DecValTok{1}\SpecialCharTok{:}\DecValTok{5}\NormalTok{,}
    \AttributeTok{labels =} \FunctionTok{c}\NormalTok{(}\StringTok{"Okuma Yazma Bilmeyen"}\NormalTok{, }\StringTok{"Ilkogretim"}\NormalTok{, }\StringTok{"Genel Lise"}\NormalTok{, }\StringTok{"Meslek Lisesi"}\NormalTok{, }\StringTok{"Yuksekogretim"}\NormalTok{)}
\NormalTok{  ) }\SpecialCharTok{+}
  \FunctionTok{labs}\NormalTok{(}\AttributeTok{title =} \StringTok{"Egitim Duzeyine Gore Istihdam Orani"}\NormalTok{,}
       \AttributeTok{x =} \StringTok{"Egitim Duzeyi"}\NormalTok{, }\AttributeTok{y =} \StringTok{"Istihdam Orani (\%)"}\NormalTok{, }\AttributeTok{color =} \StringTok{"Cinsiyet"}\NormalTok{) }\SpecialCharTok{+}
\NormalTok{  my\_theme}

\CommentTok{\# Grafik 2: İşsizlik Oranı}
\NormalTok{g2 }\OtherTok{\textless{}{-}} \FunctionTok{ggplot}\NormalTok{(veri, }\FunctionTok{aes}\NormalTok{(}\AttributeTok{x =}\NormalTok{ egitim\_sira, }\AttributeTok{y =}\NormalTok{ issizlik\_orani, }\AttributeTok{color =}\NormalTok{ cinsiyet)) }\SpecialCharTok{+}
  \FunctionTok{geom\_point}\NormalTok{(}\AttributeTok{size =} \DecValTok{3}\NormalTok{) }\SpecialCharTok{+}
  \FunctionTok{geom\_smooth}\NormalTok{(}\AttributeTok{method =} \StringTok{"lm"}\NormalTok{, }\AttributeTok{se =} \ConstantTok{FALSE}\NormalTok{, }\AttributeTok{linewidth =} \FloatTok{1.2}\NormalTok{) }\SpecialCharTok{+}
  \FunctionTok{scale\_color\_manual}\NormalTok{(}\AttributeTok{values =}\NormalTok{ my\_colors) }\SpecialCharTok{+}
  \FunctionTok{scale\_x\_continuous}\NormalTok{(}
    \AttributeTok{breaks =} \DecValTok{1}\SpecialCharTok{:}\DecValTok{5}\NormalTok{,}
    \AttributeTok{labels =} \FunctionTok{c}\NormalTok{(}\StringTok{"Okuma Yazma Bilmeyen"}\NormalTok{, }\StringTok{"Ilkogretim"}\NormalTok{, }\StringTok{"Genel Lise"}\NormalTok{, }\StringTok{"Meslek Lisesi"}\NormalTok{, }\StringTok{"Yuksekogretim"}\NormalTok{)}
\NormalTok{  ) }\SpecialCharTok{+}
  \FunctionTok{labs}\NormalTok{(}\AttributeTok{title =} \StringTok{"Egitim Duzeyine Gore Issizlik Orani"}\NormalTok{,}
       \AttributeTok{x =} \StringTok{"Egitim Duzeyi"}\NormalTok{, }\AttributeTok{y =} \StringTok{"Issizlik Orani (\%)"}\NormalTok{, }\AttributeTok{color =} \StringTok{"Cinsiyet"}\NormalTok{) }\SpecialCharTok{+}
\NormalTok{  my\_theme}

\CommentTok{\# İki grafiği birlikte göster}
\NormalTok{g1 }\SpecialCharTok{/}\NormalTok{ g2 }\SpecialCharTok{+} \FunctionTok{plot\_layout}\NormalTok{(}\AttributeTok{ncol =} \DecValTok{1}\NormalTok{)}
\end{Highlighting}
\end{Shaded}

\includegraphics{project_files/figure-pdf/unnamed-chunk-7-1.pdf}

\subsection{4. Bulgular}\label{bulgular}

Eğitim düzeyine göre kadın ve erkek için istihdam oranı ve işsizlik
oranı Şekil-3'te yer alan~ grafiklerle sunulmuştur. Grafiklerde
kullanılan eğri çizgiler, doğrusal regresyon eğilimlerini
göstermektedir.

İstihdam oranı analizine göre; eğitim düzeyinin istihdam oranı üzerinde
kadın ve erkek için anlamlı bir etkisi olduğu sonucu elde edilmektedir.
Eğitim düzeyi hem erkeklerin hem de kadınların istihdam oranını artırmak
ile birlikte, erkeklerin istihdamı için daha güçlü ve pozitif bir etki
yaratmaktadır. Kadınların eğitimden elde ettiği istihdam artışının
erkeklere göre ortalama \%8.04 daha az olduğunu görülmektedir.~ Bu
durum, cinsiyete dayalı yapısal farklılıkların eğitim düzeyine rağmen
varlığını sürdürdüğünü göstermektedir.

İşsizlik oranı analizine göre; Şekil-3'teki grafik incelendiğinde
erkekler için eğitim düzeyi arttıkça işsizlik oranının azaldığı
görülmektedir. Ancak bu, istatistiksel olarak anlamlı bir azalma
değildir. Kadınlarda eğitim düzeyi ile işsizlik oranı arasında
istatistiksel olarak anlamlı bir ilişki vardır. Kadınların eğitim düzeyi
her arttığında, işsizlik oranları erkeklere kıyasla \%6.035 daha fazla
artmaktadır. Yani kadınlarda eğitim arttıkça işsizlik oranı düşmemekte,
aksine~ artmaktadır.

Kadınlarda eğitim seviyesi arttıkça beklentiler de artıyor ama uygun iş
bulamama ihtimali yüksek olduğundan işsizlik oranı artıyor olabilir.
Ayrıca nitelikli kadın işgücünün çalışma hayatına tam olarak entegre
olamaması, cam tavan etkisi, ataerkil yapılar gibi sosyal faktörler de
bu durumu etkiliyor olabilir.

\subsection{5. Sonuç ve Öneriler}\label{sonuuxe7-ve-uxf6neriler}

Bu analiz sonucunda, eğitim düzeyinin artmasının istihdam oranını
pozitif yönde etkilediği~ tespit edilmiştir. Ayrıca, cinsiyet faktörünün
özellikle düşük eğitim düzeylerinde belirgin farklılık oluşturduğu
gözlemlenmiştir. Yükseköğretim seviyesinde kadın ve erkeklerin istihdam
oranları arasındaki fark azalmakta olup, bu durum eğitim düzeyinin
cinsiyet eşitliği üzerinde pozitif bir etkisi olduğunu göstermektedir.
Eğitim düzeyinin erkeklerde istihdam oranını anlamlı şekilde artırdığı,
kadınlarda ise eğitim artışının pozitif fakat erkeklere kıyasla daha
düşük etki yarattığı görülmektedir. Kadınların iş gücüne daha etkin
katılımı için sadece eğitim değil, aynı zamanda toplumsal cinsiyet
eşitliği politikalarının ve kadın istihdamına yönelik pozitif ayrımcılık
politikalarının da destekleyici olması gerektiği açıktır.

Bu çalışma, Türkiye'de işgücü piyasasında eğitim seviyesinin
artırılmasının kadınların istihdamını olumlu yönde etkileyeceğine dair
önemli kanıtlar sunmaktadır. Bu nedenle, eğitim politikalarının
kadınların istihdam edilebilirliğini artıracak şekilde planlanması
gerekmektedir.

Türkiye'de kadınların istihdama dahil edilmesi, işsizlik oranlarının
azaltılması ve her alanda mevcut durumlarının iyileştirilmesi konusunda
çok daha fazla politika uygulanması, yeni politikalar oluşturulması ve
en önemlisi istikrarlı bir duruş sergilenmesi gerekmektedir. Bu konuda
yapılacak en önemli adımlardan biri toplumun bilinçlendirilmesidir. Bu
amaçla kadınların çalışması kendilerinin ve ailelerinin ekonomik olarak
güçlenmesini sağlarken aynı zamanda ülkenin de gelişmişlik seviyesine
katkı sağlayacağı gerçeği topluma entegre edilmelidir.Kadınların işgücü
piyasasındaki konumlarının iyileştirilmesi için yapılabilecek bir diğer
düzenleme, politikalarda değişikliklerin yapılması ve eksikliklerin
giderilmesidir. Türkiye'de annelik ve bakım konusundaki mevcut
düzenlemelere babaların da dahil edilmesi, ücretli ebeveyn izinleri
oluşturulması ve devletin daha çok kadının üzerinde olan bakım
sorumluluğu için hizmet alanını genişletmesi gerekmektedir. Kadınların
istihdama dahil olmasını teşvik eden politikalar söz konusu olmalıdır.
Kadınların kariyerlerinde yükselmeleri için bütün kurumlarda cam tavan
engelinin ortadan kaldırılması adına belirlenecek kota ile yönetim
kademelerinde kadınlarında yer alması sağlanmalıdır.




\end{document}
